\titleformat{\chapter}[display]
  {\normalfont\Large\bfseries\centering}
  {\chaptertitlename\ \thechapter}{0pt}{\Huge}
\chapter{INTRODUCTION}
\thispagestyle{fancy}
\section{Full Stack Development}
Full-stack web development is a subset of web development that encompasses all of the responsibilities involved in creating 
websites for intranet or internet hosting.It helps to find out a perfect solution for all the front-end, testing, mobile application back-end, etc. full-stack developer work on all 
this, and also it will take care of the entire procedure of a project.
Web development, which generally refers to the activities involved in creating websites for intranet or internet hosting, includes full stack development. It involves the creation of an entire application, including the client-side (front end) and server-side (back end). The field of full stack development and how it has impacted development have undergone significant changes since the advent of cloud computing.
A software system or web development layer that includes both the front-end and back-end components of an associated application is called a full stack. The front-end of your program is what users can see and interact with. What users cannot see, such as the database, server, and logic of an application, is known as the back-end. The front-end and back-end technologies that create a website or application function well, and a full-stack web developer is at ease using both.

\section{Components of Web Development}
 Web development can be divided into three layers:
\newline
\textbf{1.Frontend Development-}  This is the part of the website that users interact with
directly. It is built using HTML, CSS, and JavaScript. Frontend frameworks and libraries like React, Angular, and Vue.js have become popular for creating sophisticated
user interfaces.[1]
\newline\textbf{2.Backend Development-} This involves the server-side of web applications, where
business logic, database interactions, and user authentication are handled. 
Back-end Development refers to server-side development.It  is the  term  used for  the  behind-the-scenes activities  that occur when performing any action on a website.
React.JS: There are so many JavaScript frameworks but we consider here only React.JS because it is easy to use than other 
frameworks like Angular.JS, Vue.js, etc. React.JS is managed by and developed by Facebook Community.[2] 
\section{Key Technologies}
\subsection*{1.HTML and CSS}The skeleton and skin of web pages. HTML structures the content,
while CSS styles it.
\subsection*{2.JavaScript} The scripting language that brings interactivity to web pages. Its ecosystem includes numerous frameworks and libraries.
\subsection*{3.Version control systems}Git is essential for managing changes to the project codebase, allowing multiple developers to work together smoothly.
\subsection*{4.APIs}Application Programming Interfaces allow different software components to communicate.
\subsection*{5.Responsive Design}Techniques that ensure websites work well on various devices,adjusting layout based on screen size.
\section{Front-end Development}The process of implementing design on the web is called front-end development. A website's pages are the culmination of its structure, data, design, content, and functionality. To put it another way, the front-end is the area of a website where users can view and interact with the command line and graphical user interface (GUI), which includes the layout, menus, text, images, videos, and more.[2].\newline The two most important types of front-end designs:
\subsection*{-User Experience (UX)}
\subsection*{-User Interface (UI)}
Although they appear to be the same, these things are distinct as we get to know them. An example of good UI but poor UX would be things like graphic components, animations, images, videos, etc. that appear beautiful on the website but are hard to make, and vice versa; a well-designed website should have an intuitive user experience that doesn't require the user to think too much.
From the programmer’s perspective front-end or the part that users see when they visit the website is mainly about the design and to make to look it good somehow. The above elements UX, UI are taken in consideration in developing web programs or web pages, apps or applications for iOS or Android, Windows or MacOS.\bigskip
\begin{itemize}
\item  HTML and CSS: HTML (HyperText Markup Language) is the backbone of any web application; it’sused to create the basic structure and content of a webpage.
\end{itemize}
\begin{itemize}
\item CSS (Cascading Style Sheets) defines the visual appearance of the HTML elements onthe screen, handling layout, design, and some types of animations.
\end{itemize}
\begin{itemize}
\item JavaScript is used to create dynamic changes on the website, enabling interactive elements such as drop-down menus, modals, forms, and even complex animations and graphics.
\end{itemize}
\begin{itemize}
\item React.js: A library developed by Facebook, React makes it easy to build interactive UIs with efficient updates and rendering of the components that have changed.
\end{itemize}
\subsection{HTML} HTML, or Hypertext Markup Language, is the standard language used to create and
structure content on web pages. It provides a set of tags and attributes that define
the structure and presentation of a document, allowing developers to create rich, in
teractive, and accessible web pages. HTML documents follow a hierarchical structure,
beginning with a root ⟨html⟩ element that contains ⟨head⟩ and ⟨body⟩ sections. The
⟨head⟩ section includes meta-information about the document, such as the title and
links to external resources, while the ⟨body⟩ section contains the visible content of the
webpage, including text, images, links, and other elements.
\subsection*{HTML Structure}  HTML documents follow a hierarchical structure that defines the content and layout
of a webpage. 
\begin{itemize}
	\item ⟨!DOCTYPE html⟩: Declaration of the document type and version (HTML5).
\end{itemize}
\begin{itemize}
	\item ⟨html⟩⟨/html⟩ - the ⟨html⟩ element: Sometimes referred to as the root element, this element encloses all the page's content. The lang property, which specifies the document's default language, is also included.[3]
\end{itemize}
\begin{itemize}
	\item ⟨head⟩⟨/head⟩ - the ⟨head⟩ element: This element serves as a container for whatever you wish to incorporate on the HTML page but isn't showing the users of your page. This comprises elements like character set declarations, keywords, and a page summary that you wish to show up in search results.[3]
\end{itemize}
\begin{itemize}
	\item ⟨meta charset="utf-8"⟩: The character set that should be used in your work is UTF-8, which contains most of the characters from most written languages. In essence, it can now handle whatever text you add to it.[3] 
\end{itemize}
\begin{itemize}
	\item ⟨meta name="viewport" content="width=device-width"⟩: By ensuring that the page renders at the viewport width, this viewport element stops mobile browsers from delivering pages that are broader than the viewport (screen) and then scaling them down.[3]
\end{itemize}
\begin{itemize}
	\item ⟨title⟩⟨/title⟩ - the ⟨title⟩ element: This determines the title of your page, which is shown in the browser tab where the page is loaded.
\end{itemize}
\begin{itemize}
	\item ⟨body⟩: Contains the content of the webpage visible to users.
\end{itemize}
\subsection{CSS} CSS, or Cascading Style Sheets, is a styling language used to control the presentation
and layout of HTML documents. It allows web developers to define the visual appear
ance of elements on a webpage, including colors, fonts, spacing, and positioning. By separating the content (HTML) from its presentation (CSS), CSS enables greater flexibility and efficiency in web design. CSS works by applying styles to HTML elements using selectors, properties, and values. Selectors target specific elements or groups of elements, while properties define the visual characteristics, and values specify the desired settings. With CSS, developers can create visually appealing, responsive, and consistent web designs across different devices and screen sizes, enhancing the overall user experience.
\subsection*{Styles} CSSconsists of a set of rules that define how HTML elements should be displayed. Each
rule consists of a selector, followed by a declaration block enclosed in curly braces {}. The declaration block contains one or more property-value pairs separated by  semicolons ;.
\subsection*{Selectors} 
Selectors are patterns used to select and style HTML elements. They can target elements based on their type, class, ID, attributes, or relationship with other elements.
\subsection*{Properties and Values} CSS properties define the visual characteristics of elements, such as color, size, font,
spacing, and positioning. Each property has a corresponding value that specifies the
desired setting.
\subsection*{Box Model} The CSS box model describes the layout and spacing of elements on a webpage. It consists of the content area, padding, border, and margin.
\subsection*{Cascade and Specificity}  CSS follows the cascade and specificity rules to determine which styles apply to an
element. The cascade determines the order of precedence for conflicting styles, while specificity determines which style rule takes precedence based on its specificity value.
\subsection*{Responsive Design}  CSS enables developers to create responsive web designs that adapt to different screen sizes and devices. Techniques such as media queries, flexbox, and grid layout help create layouts that are fluid and adaptable.
\subsection{JavaScript}  JavaScript, often abbreviated as JS, is a high-level, interpreted programming language primarily used for client-side web development. Developed by Brendan Eich in 1995, JavaScript has evolved into one of the most widely used languages for building dynamic and interactive web applications. Unlike HTML and CSS, which are markup and styling languages respectively, JavaScript is a full-fledged programming language capable of performing complex tasks and computations.JavaScript is known for its versatility and ubiquity, as it is supported by all modern web browsers and can be used across various platforms, including web servers, mobile devices, and desktop applications. It enables developers to enhance the functionality of web pages by adding interactivity, manipulating the Document Object Model (DOM), and handling events triggered by user interactions.
\subsection*{Features and Compatibility}
\begin{enumerate}
\item Dynamic Content: JavaScript allows developers to dynamically update and modify the content of web pages in real-time. This includes adding, removing, or
modifying HTML elements, changing CSS styles, and updating text and images based on user actions or external events.
\item  Event Handling: JavaScript enables developers to respond to user interactions,such as clicks, mouse movements, keyboard inputs, and form submissions, by attaching event listeners to HTML elements. This allows for the creation of interactive and responsive user interfaces.
\item DOM Manipulation: JavaScript provides access to the Document Object Model (DOM), a hierarchical representation of the structure and content of webpages. Developers can manipulate the DOM using JavaScript to dynamically update the appearance and behavior of web pages without reloading the entire page.
\item  Client-Side Validation: JavaScript can perform client-side form validation to en	sure that user input meets specified criteria before submitting data to a server.
This helps improve the user experience by providing immediate feedback and reducing the likelihood of errors.
\item Cross-Browser Compatibility: JavaScript libraries and frameworks, such as jQuery,React, Angular, and Vue.js, provide abstractions and utilities that simplify development and help address cross-browser compatibility issues, ensuring thatJavaScript code runs consistently across different web browsers and devices.
\subsection{ReactJS} React, sometimes referred to as a frontend JavaScript framework, is a JavaScript library created by Facebook.React is a tool for building UI components.he main objective of ReactJS is to develop User Interfaces (UI) that improves the speed of the apps. It uses virtual DOM (JavaScript object), which improves the performance of the app. The JavaScript virtual DOM is faster than the regular DOM. We can use ReactJS on the client and server-side as well as with other frameworks. It uses component and data patterns that improve readability and helps to maintain larger apps.
\newline A ReactJS application is made up of multiple components, each component responsible for outputting a small, reusable piece of HTML code. The components are the heart of all React applications. These Components can be nested with other components to allow complex applications to be built of simple building blocks. ReactJS uses virtual DOM based mechanism to fill data in HTML DOM. The virtual DOM works fast as it only changes individual DOM elements instead of reloading complete DOM every time.
\subsection*{DOM(Document Object Model)}DOM stands for 'Document Object Model'. It is a structured representation of HTML in the webpage or application. It represents the entire UI(User Interface) of the web application as the tree data structure.The DOM is rendered and manipulated with every change for updating the application User Interface, which affects the performance and slows it down.React maintains two virtual DOMs every time. The first one contains the updated virtual DOM, and the other is a pre-updated version of the updated virtual DOM. It compares the pre-updated version of the updated virtual DOM and finds what was changed in the DOM, like which components will be changed.
\end{enumerate}
\newpage






