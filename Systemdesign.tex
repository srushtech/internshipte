\titleformat{\chapter}[display]
  {\normalfont\Large\bfseries\centering}
  {\chaptertitlename\ \thechapter}{0pt}{\Huge}
\chapter{METHODOLOGICAL DETAILS}
\thispagestyle{fancy}
\subsection{Training} My role during the internship was to be a Fullstack web developer, and my training focused on that area. Since I had little experience in web development, I began with the basics and slowly moved on to more advanced topics. The training involved studying resources provided by the company, which included online video courses purchased from various websites. One small issue was that the company did not offer a specific learning path or a suggested order for watching the courses, which would have been helpful for beginners like me in web development. \smallskip
I found out that TypeScript is a programming language based on JavaScript. It adds features like optional type annotations, interfaces, and classes. Because of this, I thought it would be smart to understand JavaScript well before starting with TypeScript.
\newline
During my Full Stack Development internship, I adopted a methodical strategy to address a range of tasks and projects. This approach included deconstructing intricate tasks into smaller, more manageable components, utilizing industry best practices, and adhering to a systematic process for both front-end and back-end development. The following is a comprehensive overview of the methodology I employed during the internship.
\begin{enumerate}
\item \textbf{Web Development Overview:} Provides an overview of various aspects of web development, the role of the web developer, and the various technologies used in this area.
\item \textbf{Programming Fundamentals:}  Interns should introduce basic programming concepts such as variables, data structures, loops, conditions, and functions.
\item \textbf{Hands-on Coding Projects:} During the internship, students will be required
to work on projects designed to help them practice and apply the concepts learned in the training sessions. We should also work in teams whenever possible to foster collaboration and learn from each other.
\item  \textbf{Project Management and Workflow:} Teach students various project management methodologies such as Agile and Scrum and how to properly plan, execute and review projects. Career Advice: Advice on building a professional portfolio,networking and entering the industry.
\item \textbf{Final Project:}At the end of the internship, students will apply their new skills by working on a final project that will integrate what they have learned and add to their professional portfolio.
\end{enumerate}
\subsection{Internship Tasks}
\subsection*{Task 1}
\subsection*{Portfolio} Intern’s current portfolio lacks structure, visual coherence, and fails to effectively show
case their skills, projects, and achievements. As a crucial tool for personal branding and professional networking, a well-crafted portfolio website is essential to showcase the intern’s capabilities and accomplishments to potential employers and clients. Design and build a personal portfolio website showcasing the intern’s skills, projects, and achievements. The portfolio website should serve as a professional showcase of the intern’s capabilities in web development, highlighting their proficiency in coding languages, frameworks, and design principles. This task will not only allow interns to demonstrate their creativity and technical expertise but also serve as a valuable asset for future career opportunities.
\begin{itemize}
\item Problem Statement:- Individuals or businesses often struggle to showcase their work effectively to potential clients or employers.- Existing portfolios may lack organization, clarity, or visual appeal, making it difficult to convey the quality and diversity of the work.
\item  Expected Solution:- Develop a well-structured and visually engaging portfolio that effectively showcases the individual’s or business’s work, skills, and achievements. Provide clear navigation, intuitive layout, and compelling visuals to captivate visitors and encourage them to explore further.Tailor the portfolio to the target audience, whether it’s potential clients, employers, or collaborators.
\item Outputs: A polished and professional portfolio that effectively communicates the individual’s or business’s skills, expertise, and accomplishments.
\end{itemize}
\newpage
\begin{figure}[tbph]
\centering	\includegraphics[width=0.7\linewidth]{"../../../Pictures/Screenshots/Screenshot 2025-01-27 223221"}
\caption{Portfolio Page}
\label{fig:screenshot-2025-01-27-223221}
\end{figure}
\begin{figure}[tbph]
\centering
\includegraphics[width=0.7\linewidth]{"../../../Pictures/Screenshots/Screenshot 2025-01-27 223338"}
\caption{About Me}
\label{fig:screenshot-2025-01-27-223338}
\end{figure}
\begin{figure}[tbph]
\centering
\hspace{-5mm}
\vspace{-10mm}
\includegraphics[width=0.7\linewidth]{"../../../Pictures/Screenshots/Screenshot 2025-01-27 223356"}
\caption{Projects}
\label{fig:screenshot-2025-01-27-223356}
\end{figure}
\newpage
\begin{figure}[tbph]
\centering
\includegraphics[width=0.7\linewidth]{"../../../Pictures/Screenshots/Screenshot 2025-01-27 223416"}
\caption{Contact Me}
\label{fig:screenshot-2025-01-27-223416}
\end{figure}
\subsection*{Task 2}
\subsection*{Calculator}
\begin{itemize}
\item Problem Statement:- Users often encounter situations where they need to perform calculations quickly and accurately, but may lack access to specialized tools
or knowledge.- Existing calculators may be limited in functionality, difficult to use, or not tailored to specific needs or industries.
\item  Expected Solution:- Develop a user-friendly and versatile calculator that addresses common calculation needs and provides a seamless experience for users. Incorporate relevant features, functions, and customization options to meet the specific requirements of the target audience.Ensure the calculator is intuitive,reliable, and accessible across various devices and platforms.
\item Outputs:- A fully functional calculator application or tool that meets the needs and expectations of the target audience.
\end{itemize}
\newpage
\begin{figure}[tbph]
\centering
\includegraphics[width=0.7\linewidth]{"../../../.vscode/my repo/Srushti/images/calc"}
\caption{Calculator}
\label{fig:calc}
\end{figure}
\begin{figure}[tbph]
\centering
\hspace{-5mm}
\vspace{-10mm}
\includegraphics[width=0.7\linewidth]{"../../../Pictures/Screenshots/Screenshot 2025-01-27 233010"}
\caption{Calculator}
\label{fig:screenshot-2025-01-27-233010}
\end{figure}
\subsection*{Task 3}
\subsection*{Weather App}
A weather app gives users up-to-date weather information for a chosen place, including details like temperature, humidity, wind speed, and overall weather. It uses HTML, CSS, JavaScript, and APIs to get live data from a weather service and shows it in a simple format.This project focuses on creating a web-based weather app where users can enter a city or location, and the app will show the current weather. The app uses HTML for layout, CSS for design, JavaScript for user interaction and API calls, and an external weather API (such as OpenWeatherMap or WeatherStack) to gather the data.
\begin{itemize}
\item Problem Statement:-Many people think it takes too long to look up weather information on different platforms. They often need to check various websites or apps to see the current weather for a specific place. Also, many of these sources are not user-friendly, which makes it difficult for users to find the information they need quickly.
\item Expected Solution:- A simple input box where users can enter a city or place name to get the weather information.  Gets Live Weather Data: The app will connect to a weather service (like OpenWeatherMap or WeatherStack) to get up-to-date weather data for the entered location.  
Shows Weather Details: The app will show the current weather, including temperature, wind speed, humidity, and overall weather conditions.
\item Outputs: The Expected outputs includes,
\end{itemize}
\begin{figure}[tbph]
	\centering
	\includegraphics[width=0.7\linewidth]{"../../../Pictures/Screenshots/Screenshot 2025-01-27 233245"}
	\caption{Weather App}
	\label{fig:screenshot-2025-01-27-233245}
\end{figure}
\newpage
\begin{figure}[tbph]
	\centering
	\includegraphics[width=0.7\linewidth]{"../../../Pictures/Screenshots/Screenshot 2025-01-27 233255"}
	\caption{Weather App}
	\label{fig:screenshot-2025-01-27-233255}
\end{figure}
\subsection*{Task 4}
\subsection*{To-do list Using ReactJS}
\begin{enumerate}
\item Problem Statement:-In today's busy world, being organized and handling tasks well is essential for getting things done. Many people find it hard to keep track of their daily tasks and often forget important ones. Regular to-do lists, whether on paper or in simple apps, usually don't have the features needed for effective task management.
Inefficiency: Basic to-do lists are fixed and often need manual changes, which can make them less engaging to use.
Task Management Challenges: Users may struggle to quickly prioritize, complete, or remove tasks.Unsatisfactory User Experience: Many to-do list apps do not offer real-time updates, leading to the need for page refreshes and slow task handling.
\item Expected Solution:- Create New Tasks: Users can enter a task in a text box and press a button to include it in the list. The tasks will be saved in the app's memory.  
Complete Tasks: Each task will feature a checkbox that users can select to indicate it is done. When checked, the task will change visually (like being crossed out or highlighted).  
Remove Tasks: Each task will have a delete button that lets users take it off the list when it is not needed anymore.
\item Outputs:- After completing tasks user will experience following outputs,
\end{enumerate}
\newpage
\begin{figure}[tbph]
\centering
\includegraphics[width=0.7\linewidth]{"../../../Pictures/Screenshots/Screenshot 2025-01-27 235844"}
\caption{Modern To-do List}
\label{fig:screenshot-2025-01-27-235844}
\end{figure}
\begin{figure}[tbph]
\centering
\includegraphics[width=0.7\linewidth]{"../../../Pictures/Screenshots/Screenshot 2025-01-27 235914"}
\caption{Modern To-do List}
\label{fig:screenshot-2025-01-27-235914}
\end{figure}
\newpage
\subsection{Real-World Project}
\subsection*{Caygnus Mocks:A Professional Platform for Mock Tests}
Caygnus Mocks is a modern and comprehensive online platform designed to assist students in preparing for competitive exams such as JEE, NEET, JEE-Mains, and other similar examinations. The platform provides a professional exam environment where students can practice mock MCQ-based questions, assess their performance, and track their preparation progress effectively.
\subsection*{Objectives}
The primary goal of Caygnus Mocks is to offer students a realistic and reliable mock test experience to enhance their readiness for competitive exams. By simulating professional exam settings, the platform aims to:
\begin{enumerate}
	\item Familiarize students with the actual exam pattern and time management.
	\item Provide a vast repository of curated MCQ questions tailored to the syllabus.
	\item Deliver instant performance insights and scores to help students identify strengths and weaknesses.
\end{enumerate}
\subsection*{Features and Functionalities}
\begin{enumerate}
\item Mock Tests:
	\begin{itemize}
\item Exams for JEE, NEET, and other competitive exams.
\item Timer-based tests with a user-friendly interface.
\item Dynamic question sets categorized by subject and difficulty.
	\end{itemize}
\item Performance Tracking:
\begin{itemize}
\item Detailed score reports for each mock test.
\item Analytics showing strong and weak subject areas.
\item Performance trends to monitor improvement over time.
\end{itemize}
\item Professional Exam Simulation:
\begin{itemize}
\item Realistic user interface mimicking actual competitive exams.
\item Adherence to exam durations and structures.
\item A stress-free environment to practice under realistic conditions.
\end{itemize}
\item User Accessibility:
\begin{itemize}
\item Easy registration and login process for students.
\item Mobile-friendly design for seamless access on all devices.
\item Secure database to store and retrieve user progress data.
\end{itemize}
\end{enumerate}
\subsection*{Technology Stack:}
\begin{itemize}
\item Frontend: React.js, Tailwind CSS for a responsive and modern UI.
\item Backend: Node.js with Firebase for data management and authentication.
\item Database: Firestore for storing user data and test scores.
\item Hosting: Deployed on a scalable cloud platform to ensure high performance and availability.
\end{itemize}
By bridging the gap between preparation and real exam experiences, Caygnus Mocks aims to become a go-to platform for students aspiring to succeed in competitive exams. Its commitment to quality, usability, and performance ensures that students are well-prepared to achieve their academic goals.
\begin{figure}[tbph]
	\centering
	\includegraphics[width=0.9\linewidth]{../../caygnus-mocks}
	\caption{Hero Section}
	\label{fig:caygnus-mocks}
\end{figure}
\newpage
\begin{figure}[tbph]
	\centering
	\includegraphics[width=0.9\linewidth]{../../caynus2}
	\caption{Hero section}
	\label{fig:caynus2}
\end{figure}
\begin{figure}[tbph]
	\centering
	\includegraphics[width=0.9\linewidth]{../../caygnus3}
	\caption{Test}
	\label{fig:caygnus3}
\end{figure}
\begin{figure}[tbph]
	\centering
	\includegraphics[width=0.9\linewidth]{../../cayg4}
	\caption{Test}
	\label{fig:cayg4}
\end{figure}
\begin{figure}[tbph]
	\centering
	\includegraphics[width=0.9\linewidth]{../../cayg5}
	\caption{Test}
	\label{fig:cayg5}
\end{figure}
\begin{figure}[tbph]
	\centering
	\includegraphics[width=0.9\linewidth]{../../cayg6}
	\caption{Test}
	\label{fig:cayg6}
\end{figure}
\begin{figure}[tbph]
	\centering
	\includegraphics[width=0.9\linewidth]{../../cayg7}
	\caption{Test}
	\label{fig:cayg7}
\end{figure}
\newpage
\begin{figure}[tbph]
	\centering
	\includegraphics[width=0.9\linewidth]{../../cayg8}
	\caption{Test}
	\label{fig:cayg8}
\end{figure}







